\documentclass[a4paper,11pt]{report}
 
\usepackage[utf8]{inputenc}  
\usepackage[T1]{fontenc}
\usepackage[francais]{babel}
\usepackage{amsfonts}
\usepackage{amsmath}
\usepackage{listings}
\usepackage{fullpage}
\usepackage{fancyhdr}
\pagestyle{fancy}

\renewcommand{\thesection}{\arabic{section}}
\renewcommand{\thesubsection}{\roman{subsection}}

\renewcommand{\headrulewidth}{0pt}
\fancyhead[C]{} 
\fancyhead[L]{}
\fancyhead[R]{}

\renewcommand{\footrulewidth}{1pt}
\fancyfoot[C]{\textbf{\thepage}} 
\fancyfoot[L]{Delplanque Julien}
\fancyfoot[R]{2013-2014}

\begin{document}
\renewcommand{\labelitemi}{$\cdot$}
\begin{Large}\begin{center} 
   \underline{\textbf{Analyse I (Partie B): Propriétés sur les petits o}} 
\end{center}\end{Large}
%%%%%
\section{Définition de petit o}
Soit $f:\mathbb{R}\rightarrow\mathbb{R}$, $a\in Dom(f)$, on dit que f est un petit o de $(x-a)^n$ ssi\\
\begin{center}
	$f(a)=0$ et $\lim\limits_{x \rightarrow a}{\frac{f(x)}{(x-a)^n}}=0$
\end{center}

%%%%%
\section{Propriétés}
%%%
\subsection{Soit $m,n \in \mathbb{N}$, si $ m \le n$, alors $o((x-a)^n) = o((x-a)^m)$}
Soit $f(x) = o((x-a)^n)$,\\
Donc par définition de petit o,
\begin{center}
	$f(a) = 0$ et $\lim\limits_{x \rightarrow a}{\frac{f(x)}{(x-a)^n} = 0}$
\end{center}
Montrons que $f(x) = o((x-a)^m)$ c'est-à-dire montrons que $f(a) = 0$ et $\lim\limits_{x \rightarrow a}{\frac{f(x)}{(x-a)^m}} = 0$
\begin{center}
\begin{tabular}{rcll}
$f(a)$ & $=$ & $0$ & par hypothèse\\
&&& \\
$\lim\limits_{x \rightarrow a}{\frac{f(x)}{(x-a)^m}}$ & $=$ & $\lim\limits_{x \rightarrow a}{\frac{f(x)}{(x-a)^n (x-a)^{m-n}}}$ & \\
& $=$ & $\lim\limits_{x \rightarrow a}{\frac{f(x)}{(x-a)^n}}.\lim\limits_{x \rightarrow a}{\frac{1}{(x-a)^{m-n}}}$ & car la limite du produit est le produit des limites\\
& $=$ & $0.\lim\limits_{x \rightarrow a}{(x-a)^{n-m}}$ & par hypothèse et car $m \le n$ donc $n-m \ge 0$\\
& $=$ & $0.0$ & car $x \rightarrow a$ \\
& $=$ & 0 & \\
\end{tabular}
\end{center}
Cqfd.

%%%
\subsection{Soit $m,n \in \mathbb{N}$, $o((x-a)^n).o((x-a)^m) = o((x-a)^{n+m})$}
Soit $f(x) = o((x-a)^n$,

Soit $g(x) = o((x-a)^m$,\\
Donc par définition de petit o,
\begin{center}
	$f(a) = 0$ et $\lim\limits_{x \rightarrow a}{\frac{f(x)}{(x-a)^n} = 0}$\\
	$g(a) = 0$ et $\lim\limits_{x \rightarrow a}{\frac{f(x)}{(x-a)^m} = 0}$
\end{center}

Posons $h(x) = f(x).g(x)$,\\
Montrons que $h(x) = o((x-a)^{n+m})$ c'est à dire montrons que $h(a) = 0$ et $\lim\limits_{x \rightarrow a}{\frac{h(x)}{(x-a)^{n+m}}} = 0$
\begin{center}
\begin{tabular}{rcll}
$h(a)$ & $=$ & $f(a).g(a)$ & par définition de $h$\\
& $=$ & $0.0$ & par hypothèse \\
& $=$ & $0$ &\\
&&& \\
$\lim\limits_{x \rightarrow a}{\frac{h(x)}{(x-a)^{n+m}}}$ & $=$ & $\lim\limits_{x \rightarrow a}{\frac{h(x)}{(x-a)^n (x-a)^{m}}}$ & \\
& $=$ & $\lim\limits_{x \rightarrow a}{\frac{f(x)}{(x-a)^n}}.\lim\limits_{x \rightarrow a}{\frac{g(x)}{(x-a)^m}}$ & car la limite du produit est le produit des limites\\
& $=$ & $0.0$ & par hypothèse\\
& $=$ & $0$ & \\
\end{tabular}
\end{center}
Cqfd.

%%%
\subsection{Soit $m,n \in \mathbb{N}$, $(o((x-a)^n))^m = o((x-a)^{n.m})$}
Soit $f(x) = o((x-a)^n)$,\\
Donc par définition de petit o,
\begin{center}
	$f(a) = 0$ et $\lim\limits_{x \rightarrow a}{\frac{f(x)}{(x-a)^n} = 0}$
\end{center}

Posons $h(x) = (f(x))^m$,\\
Montrons que $h(x) = o((x-a)^{n.m})$ c'est à dire montrons que $h(a) = 0$ et $\lim\limits_{x \rightarrow a}{\frac{h(x)}{(x-a)^{n.m}}} = 0$

\begin{center}
\begin{tabular}{rcll}
$h(a)$ & $=$ & $(f(a))^m$ & par définition de $h$\\
& $=$ & $(0)^m$ & par hypothèse \\
& $=$ & $0$ &\\
&&& \\
$\lim\limits_{x \rightarrow a}{\frac{h(x)}{(x-a)^{n.m}}}$ & $=$ & $\lim\limits_{x \rightarrow a}{\frac{(f(x))^m}{(x-a)^{n.m}}}$ & par définition de $h$\\
& $=$ & $\lim\limits_{x \rightarrow a}{\frac{(f(x))^m}{((x-a)^n)^m}}$ & par une propriété sur les exposants\\
& $=$ & $\lim\limits_{x \rightarrow a}{(\frac{f(x)}{(x-a)^n})^m}$ & par une propriété sur les exposants\\
& $=$ & $(\lim\limits_{x \rightarrow a}{\frac{f(x)}{(x-a)^n}})^m$ & car la limite du produit est le produit des limites\\
& $=$ & $0^m$ & par hypothèse\\
& $=$ & $0$ & \\
\end{tabular}
\end{center}
Cqfd.

%%%
\subsection{Soit $n \in \mathbb{N}$, $o((x-a)^n) = o(1).(x-a)^n$}
Soit $f(x) = o((x-a)^n)$,\\
Donc par définition de petit o,
\begin{center}
	$f(a) = 0$ et $\lim\limits_{x \rightarrow a}{\frac{f(x)}{(x-a)^n} = 0}$
\end{center}

\begin{itemize}
	\item \textbf{si $x \ne a$} alors prenons $g(x) = \frac{f(x)}{(x-a)^n}$\\
	$g(x)$ est un petit o de 1 car:
	\begin{center}
	$g(a) = \frac{f(a)}{(x-a)^n} = \frac{0}{(x-a)^n} = 0$ et $\lim\limits_{x \rightarrow a}{\frac{g(x)}{1} = \lim\limits_{x \rightarrow a}{\frac{f(x)}{(x-a)^n}} = 0}$\\
	\end{center}
	On a bien:\\
	\begin{center}
	\begin{tabular}{rcll}
	$f(x)$ & $=$ & $g(x).(x-a)^n$ &\\
	& $=$ & $\frac{f(x)}{(x-a)^n}.(x-a)^n$ & \\
	& $=$ & $f(x)$ & \\
	\end{tabular}
	\end{center}
	Cqfd.\\
	
	\item \textbf{si $x = a$} alors prenons $g(x) = 0$
	\begin{center}
	$g(a) = 0$ et $\lim\limits_{x \rightarrow a}{\frac{g(x)}{1} = 0}$\\
	\end{center}
	On a bien:
		\begin{center}
	\begin{tabular}{rcll}
	$f(x)$ & $=$ & $g(x).(x-a)^n$ &\\
	& $=$ & $0.(x-a)^n$ & par définition de g(x) et car $x = a$\\
	& $=$ & $0$ & \\
	\end{tabular}
	\end{center}
	Cqfd.
\end{itemize}

%%%
\subsection{Soit $n \in \mathbb{N}$, $o(1).(x-a)^n = o((x-a)^n)$}
Soit $f(x) = o(1)$,\\
Donc par définition de petit o,
\begin{center}
	$f(a) = 0$ et $\lim\limits_{x \rightarrow a}{\frac{f(x)}{1} = 0}$
\end{center}

Posons $h(x) = f(x).(x-a)^n$,\\
Montrons que $h(x) = o((x-a)^n)$ c'est à dire montrons que $h(a) = 0$ et $\lim\limits_{x \rightarrow a}{\frac{h(x)}{(x-a)^{n}}} = 0$
\begin{center}
\begin{tabular}{rcll}
$h(a)$ & $=$ & $f(a).(x-a)^n$ & par définition de $h$\\
& $=$ & $0.(x-a)^n$ & par hypothèse \\
& $=$ & $0$ &\\
&&& \\
$\lim\limits_{x \rightarrow a}{\frac{h(x)}{(x-a)^{n}}}$ & $=$ & $\lim\limits_{x \rightarrow a}{\frac{f(x).(x-a)^{n}}{(x-a)^{n}}}$ & par définition de $h$\\
& $=$ & $\lim\limits_{x \rightarrow a}{f(x)}$ & \\
& $=$ & $0$ & par hypothèse\\
\end{tabular}
\end{center}
Cqfd.
\end{document}